
\documentclass{beamer}
\mode<presentation>
\usepackage{amsmath}
\usepackage{amssymb}
%\usepackage{advdate}
\usepackage{graphicx}
\graphicspath{{../figs/}}
\usepackage{adjustbox}
\usepackage{subcaption}
\usepackage{enumitem}
\usepackage{multicol}
\usepackage{mathtools}
\usepackage{listings}
\usepackage{url}
\def\UrlBreaks{\do\/\do-}
\usetheme{Boadilla}
\usecolortheme{lily}
\setbeamertemplate{footline}
{
  \leavevmode%
  \hbox{%
  \begin{beamercolorbox}[wd=\paperwidth,ht=2.25ex,dp=1ex,right]{author in head/foot}%
    \insertframenumber{} / \inserttotalframenumber\hspace*{2ex} 
  \end{beamercolorbox}}%
  \vskip0pt%
}
\setbeamertemplate{navigation symbols}{}
\let\solution\relax
\usepackage{gvv}
\lstset{
%language=C,
frame=single, 
breaklines=true,
columns=fullflexible
}

\numberwithin{equation}{section}
\title{12.703}
\author{AI25BTECH11001 - ABHISEK MOHAPATRA}
\begin{document}
{\let\newpage\relax\maketitle}
\renewcommand{\thefigure}{\theenumi}
\renewcommand{\thetable}{\theenumi}


	 	\textbf{Question}:
		Let $\vec{V} = {\vec{p} (x) = a_0 + a_1x + a_2x^2 : a_i \epsilon R}$. Define $\vec{T} : V \rightarrow V$ by
\begin{align}
		\vec{T}(\vec{p}) = (\vec{p}(0) - \vec{p}(1)) + (\vec{p}(0) + \vec{p}(1)) x + \vec{p}(0) x^2 .
\end{align}
Then the sum of eigenvalues of $\vec{T}$ equals.


\textbf{Solution:}
Given,
\begin{align}
		\vec{T}(\vec{p}) = (\vec{p}(0) - \vec{p}(1)) + (\vec{p}(0) + \vec{p}(1)) x + \vec{p}(0) x^2 .
\end{align}
Or,
\begin{align}
		\vec{T}\vec{p} = \myvec{(\vec{p}(0) - \vec{p}(1)) \\ (\vec{p}(0) + \vec{p}(1)) x \\ \vec{p}(0) x^2 }.
\end{align} where $\vec{p} = \myvec{a_0\\a_1x\\a_2x^2}$.


So, sum of the eigenvalues of T is $trace(\vec{T})$.

a solution can seen is
\begin{align}
		\myvec{0&-\frac{1}{x}&-\frac{1}{x^2}\\2x&1&\frac{1}{x}\\x^2&0&0}\vec{p} = \myvec{(\vec{p}(0) - \vec{p}(1)) \\ (\vec{p}(0) + \vec{p}(1)) x \\ \vec{p}(0) x^2 }.
\end{align}
So the sum of eigenvalues = 1.

\end{document}



